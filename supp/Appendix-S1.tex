\documentclass[12pt]{article}

\usepackage[vmargin=1in,hmargin=1in]{geometry}
\usepackage{amsmath}
\usepackage[parfill]{parskip}
\usepackage{hyperref}
\usepackage{natbib}
\usepackage{bm}
\usepackage{amsfonts}
\usepackage{graphicx}
\usepackage{abstract}
\usepackage{lineno}
\usepackage{setspace}

\hypersetup{pdfstartview={Fit},hidelinks}


\usepackage{caption}
\captionsetup[figure]{labelformat=empty}% redefines the caption setup of the figures environment in the beamer class.


\title{Ecological Applications Appendix S1 \\ Simulation study accompanying the paper: \\ \it Monitoring partially-marked populations using camera and telemetry data }
\author{Lydia L. S. Margenau$^{1}$\footnote{Corresponding author: lydia.margenau@wisconsin.gov}, Michael J. Cherry$^2$,  Karl V. Miller$^1$, \\ Elina P. Garrison$^3$, Richard B. Chandler$^1$}


\begin{document}



\maketitle

\vspace{12pt}

\begin{description}%[labelindent=1pt]%[leftmargin=1cm]%,labelwidth=\widthof{\bfseries Example:}]
%  \large
\item[$^1$] Warnell School of Forestry and Natural Resources, University of Georgia %\\
\item[$^2$] Caesar Kleberg Wildlife Research Institute at Texas A\&M University-Kingsville %\\
\item[$^3$] Florida Fish and Wildlife Conservation Commission %\\
\end{description}

\clearpage

\section*{Introduction and Methods}

We conducted a small simulation study to compare the performance of
the joint generalized spatial mark-resight (gSMR) model with the
two-stage gSMR model. One hundred datasets were simulated from the
model with parameters chosen to resemble the deer study described in
the manuscript. Specifically, parameters were $\beta_0=1$,
$\beta_1=-0.02$, $\alpha=0.5$, $\sigma_{\epsilon}=0.1$,
$\lambda_0=0.07$, and $\sigma=400$. For the joint model, we assumed a
uniform marking process with $p^{\rm cap}=0.2$. The design involved 60
cameras with the same spatial arrangement as the cameras at our AL
study site. We modeled the population over 20 primary sampling
periods, each corresponding to a 14-day period. We simulated one
telemetry location per day. Code to reproduce the simulation study can
be found at \url{https://github.com/rbchan/monitor-cam-telem}.


\section*{Results and Discussion}

Average population size in the simulation study declined from 73 to 45
during the 20 year period (Figure S1). The number of marked
individuals ranged from 1 to 27 and was approximately 20\% of the
total populatoin size over the 20 primary sampling periods (Figure S1). 

Parameter estimates (posterior means) were similar for the
encounter rate parameters (Figure S2), the density trend
parameters (Figure S3), and for the realized values of abundance
(Figure S4). 

Obtaining 10,000 MCMC draws (after discarding 2000 burn-in) took
approximately 15 hours for the two-stage model and 17 hours for the
joint model. Unlike the two-stage model, run time for the joint model
will increase with the complexity of the marking process.


\clearpage

\begin{figure}[h!]
  \centering
  \includegraphics[width=\textwidth]{sim/sim-N-n}
  \caption{Figure S1. Population size ($N$) and the number of marked
    individuals over time. }   
  \label{fig:sim-N-n}
\end{figure}



\clearpage

\begin{figure}[h!]
  \centering
  \includegraphics[width=0.7\textwidth]{sim/sim-lam0sig}
  \caption{Figure S2. Results from the 100 simulated datasets used to
    compare posterior means for the encounter rate parameters
    ($\lambda_0$ and $\sigma$) for the joint model and the two-stage
    model. }   
  \label{fig:sim-lam0sig}
\end{figure}


\clearpage

\begin{figure}[h!]
  \centering
  \includegraphics[width=0.7\textwidth]{sim/sim-betas}
  \caption{Figure S3. Results from the 100 simulated datasets used to
    compare posterior means for the density trend parameters
    ($\beta_0$ and $\beta_1$) for the joint model and 
    the two-stage model. }  
  \label{fig:sim-betas}
\end{figure}


\clearpage

\begin{figure}[h!]
  \centering
  \includegraphics[width=\textwidth]{sim/sim-Nt}
  \caption{Figure S4. Results from the 100 simulated datasets used to
    compare posterior means for abundance over time
    ($N_t$) for the joint model and the two-stage model. }  
  \label{fig:sim-N}
\end{figure}



\end{document}


